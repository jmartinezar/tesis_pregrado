\usepackage{xr}

\externaldocument{simetrias_de_gauge}

\chapter{El mecanismo de Higgs}

\section{Rompimiento de simetría en un campo real}

Considerando un campo escalar $\phi$ con el potencial $V(\phi)$

$$ V(\phi) = \frac{1}{2}\mu^2\phi^2 + \frac{1}{4}\lambda \phi^4 $$

el lagrangiano es dado por

$$ \mathcal{L} = \frac{1}{2}(\partial_{\mu}\phi)(\partial^{\mu}\phi) - V(\phi) $$

\begin{equation}
    \mathcal{L} = \frac{1}{2}(\partial_{\mu}\phi)(\partial^{\mu}\phi) - \frac{1}{2}\mu^2\phi^2 - \frac{1}{4}\lambda \phi^4 
    \label{mas1}
\end{equation}

el primer término corresponde al componente cinético, el segundo corresponde al término de la masa y el tercero corresponde a la autointeracción del campo consigo mismo. Para $V(\phi)$ se tiene que el término dominante para grandes valores de $\phi$ es $\frac{1}{4}\lambda\phi^4$, además se tiene que el término $\frac{1}{4}\phi^4>0$ por lo que $V(\phi)$ tiende a $+\infty$ o $-\infty$ cuando $\phi\rightarrow \pm\infty$ si $\lambda$ es positivo o negativo, respectivamente. Por esto, a fin de obtener un mínimo, $\lambda$ debe ser positivo. Derivando $V(\phi)$ respecto a $\phi$ e igualando a cero

$$ \frac{dV(\phi)}{d\phi} = \mu^2\phi + \lambda \phi^3 =\phi(\mu^2 + \lambda \phi^2) = 0 \rightarrow \phi = \pm \sqrt{-\frac{\mu^2}{\lambda}} \quad \text{\'o} \quad \phi = 0$$

al resultar una ecuación de tercer grado tenemos una de dos posibilidades: la ecuación tiene una solución real o la ecuación tiene tres soluciones reales. Para el primer caso, que ocurre cuando $\mu^2>0$, la única solución i.e. el punto crítico de la función $V(\phi)$ ocurre en el caso $\phi = 0$. Para el segundo caso, que ocurre cuando $\mu^2<0$, existen dos soluciones adicionales que ocurren para los casos $\pm\sqrt{-\mu^2/\lambda}$, como se muestra en la figura \ref{pot}. La elección de un valor distinto de 0 para el estado base rompe la simetría presente en el potencial.

\begin{figure}[!h]
    \centering
    \begin{subfigure}[b]{0.45\textwidth}
        \centering
        \includegraphics[width=\linewidth]{figures/pot_1.pdf}
        \caption{Para $\lambda > 0$ y $\mu^2>0$, $V'(\phi)=0$ tiene solamente una solución real: $\phi=0$}
        \label{pot1}
    \end{subfigure}
    \hfill
    \begin{subfigure}[b]{0.45\textwidth}
        \centering
        \includegraphics[width=\linewidth]{figures/pot_2.pdf}
        \caption{Para  $\lambda > 0$ y $\mu^2<0$, $V'(\phi)=0$ tiene dos soluciones adiconales a $\phi=0$ que son $\phi = \pm\sqrt{-\mu^2/\lambda}$. }
        \label{pot2}
    \end{subfigure}
    \caption{La figura ilustra la forma que toma $V(\phi)$ en el caso en que $\lambda>0$. Con $\mu^2>0$ para \ref{pot1} y con $\mu^2<0$ para \ref{pot2} }
    \label{pot}
\end{figure}

Sin pérdida de generalidad, se toma el valor de estado base como $\phi=v$. Las excitaciones del estado base pueden ser entonces representadas de la forma $\phi(x) = v+\eta(x)$. Como $v$ es constante se tiene entonces $\partial_{\mu}\phi = \partial_{\mu}\eta$. Por lo que el lagrangiano expresado en términos de la función $\eta(x)$ es

$$\mathcal{L}(\eta) = (\partial_{\mu}\eta)(\partial^{\mu}\eta) -V(\eta)$$

$$\mathcal{L}(\eta) = (\partial_{\mu}\eta)(\partial^{\mu}\eta) - \frac{1}{2}\mu^2(v+\eta(x))^2 - \frac{1}{4}\lambda(v+\eta(x))^4$$

como en el mínimo se tiene $\mu^2=-\lambda v^2$, se remplaza y expanden las potencias

$$\mathcal{L}(\eta) = (\partial_{\mu}\eta)(\partial^{\mu}\eta) - \frac{1}{2}\lambda v^2(v^2 + 2v\eta + \eta^2) - \frac{1}{4}\lambda(v^4+4v^3\eta+6v^2\eta^2+4v\eta^3+\eta^4)$$

$$\mathcal{L}(\eta) = (\partial_{\mu}\eta)(\partial^{\mu}\eta) + \frac{1}{2}\lambda v^4 + \lambda v^3\eta + \frac{1}{2}\lambda v^2\eta^2 - \frac{1}{4}\lambda v^4-\lambda v^3\eta-\frac{3}{2}\lambda v^2\eta^2-\lambda v\eta^3-\frac{1}{4}\lambda\eta^4$$

$$\mathcal{L}(\eta) = (\partial_{\mu}\eta)(\partial^{\mu}\eta)   -\lambda v^2\eta^2-\lambda v\eta^3-\frac{1}{4}\lambda\eta^4+ \frac{1}{4}\lambda v^4$$

el segundo término corresponde a la masa, con $m=\sqrt{2\lambda v^2}$, por lo que el lagrangiano representa un campo escalar masivo. El tercer y cuarto término corresponden a la autointeracción del campo $\eta$, mostrada en la figura \ref{diag1}. Mientras que el término $\frac{1}{4}\lambda v^4$ es una constante y, por tanto, no conlleva consecuencias físicas.

\begin{figure}[htbp]
    \centering
    \begin{subfigure}[b]{0.45\textwidth}
        \centering
        \begin{tikzpicture}
        \begin{feynman}
            \vertex [dot, thick, minimum size=1.5mm] (a) {};  % Vertex point enlarged
            \vertex [left=1cm of a] (label) {\(\lambda v\)}; % Label to the side of the vertex
            \vertex [above left=of a] (i1) {\(\eta\)};
            \vertex [below left=of a] (i2) {\(\eta\)};
            \vertex [right=of a] (f1) {\(\eta\)};

            \diagram* {
                (i1) -- [scalar] (a) -- [scalar] (f1),
                (i2) -- [scalar] (a),
            };
        \end{feynman}
        \end{tikzpicture}
    \end{subfigure}
    \hfill
    \begin{subfigure}[b]{0.45\textwidth}
        \centering
        \begin{tikzpicture}
        \begin{feynman}
            \vertex [dot, thick, minimum size=1.5mm] (b) {};  % Vertex point enlarged
            \vertex [left=1cm of b] (label) {\(\frac{1}{4}\lambda\)}; % Label to the side of the vertex
            \vertex [above left=of b] (i1) {\(\eta\)};
            \vertex [below left=of b] (i2) {\(\eta\)};
            \vertex [above right=of b] (i3) {\(\eta\)};
            \vertex [below right=of b] (i4) {\(\eta\)};

            \diagram* {
                (i1) -- [scalar] (b),
                (i2) -- [scalar] (b),
                (i3) -- [scalar] (b),
                (i4) -- [scalar] (b),
            };
        \end{feynman}
        \end{tikzpicture}
    \end{subfigure}
    \caption{Diagramas correspondientes a la autointeracción del campo $\eta$. A la izquierda se encuentra el diagrama correspondiente a $\eta\eta\rightarrow\eta$, mientras que a la derecha se encuentra el diagrama correspondiente a $\eta\eta\rightarrow\eta\eta$}
    \label{diag1}
\end{figure}

\section{Rompimiento de simetría en un campo complejo}

Para un campo escalar complejo dado por

$$ \phi = \frac{1}{\sqrt{2}}(\phi_1 + i\phi_2) $$

con $\phi_1$ y $\phi_2$ campos escalares reales, el lagrangiano viene dado por

$$ \mathcal{L} = (\partial^{\mu}\phi)^*(\partial_{\mu}\phi) - V(\phi)$$

con $V(\phi)=\mu^2\phi^*\phi + \lambda (\phi^*\phi)^2$. Expresando el lagrangiano en términos de los campos reales

\begin{equation}
\mathcal{L} = \frac{1}{2} (\partial^{\mu}\phi_1)(\partial_{\mu}\phi_1) + \frac{1}{2} (\partial^{\mu}\phi_2)(\partial_{\mu}\phi_2) - \frac{1}{2}\mu^2 (\phi_1^2+\phi_2^2) - \frac{1}{4}\lambda (\phi_1^2+\phi_2^2)^2
\label{rs1}
\end{equation}

De igual forma que con el anterior caso, se tiene que para que la función $\phi$ tenga un mínimo $\lambda$ debe ser mayor que cero y para que se presenten mínimos distintos de $\phi = 0$ $\mu^2$ debe ser menor que cero. En dado caso, se tiene que la función $\phi$ tiene infinitos mínimos que cumplen la condición

$$ \phi_1^2 + \phi_2^2 = v^2 $$

donde $v^2=- \mu^2/\lambda$. Por lo que el estado base corresponde a un valor de $\phi$ que cumple dicha condición. Sin pérdida de generalidad, es posible escoger el valor del estado base sobre el eje real, tal que $\phi = (v,0)$. El campo escalar complejo puede ser expandido como perturbaciones del estado base tal que $\phi_1 = v + \eta(x)$ y $\phi_2=\xi(x)$, teniendo $\phi = v + \eta(x)+i\xi(x)$. Como $v$ es constante, $\partial_{\mu}\phi_1=\partial_{\mu}\eta$ y $\partial_{\mu}\phi_2 = \partial_{\mu}\xi$. Introduciendo esta representación de $\phi$ en la ecuación \ref{rs1}

$$ \mathcal{L} = \frac{1}{2} (\partial^{\mu}\eta)(\partial_{\mu}\eta) + \frac{1}{2} (\partial^{\mu}\xi)(\partial_{\mu}\xi) - \frac{1}{2}\mu^2 ((v+\eta)^2+\xi^2) - \frac{1}{4}\lambda ((v+\eta)^2+\xi^2)^2 $$

expandiendo los cuadrados y tomando $\mu^2=-\lambda v^2$

$$ \mathcal{L} = \frac{1}{2} (\partial^{\mu}\eta)(\partial_{\mu}\eta) + \frac{1}{2} (\partial^{\mu}\xi)(\partial_{\mu}\xi) + \frac{1}{2}\lambda v^2 (v^2 + 2v\eta + \eta^2 +\xi^2) - \frac{1}{4}\lambda (v^2 + 2v\eta + \eta^2+\xi^2)^2 $$

\begin{multline}
 \mathcal{L} = \frac{1}{2} (\partial^{\mu}\eta)(\partial_{\mu}\eta) + \frac{1}{2} (\partial^{\mu}\xi)(\partial_{\mu}\xi) + \left(\frac{1}{2}\lambda v^4 + \lambda v^3\eta + \frac{1}{2}\lambda v^2\eta^2 +\frac{1}{2}\lambda v^2\xi^2\right)\\
- \frac{1}{4}\lambda \left((v^2 + 2v\eta)^2 + 2(v^2 + 2v\eta)(\eta^2+\xi^2) + (\eta^2+\xi^2)^2 \right)
\end{multline}

\begin{multline}
 \mathcal{L} = \frac{1}{2} (\partial^{\mu}\eta)(\partial_{\mu}\eta) + \frac{1}{2} (\partial^{\mu}\xi)(\partial_{\mu}\xi) + \frac{1}{2}\lambda v^4 + \lambda v^3\eta + \frac{1}{2}\lambda v^2\eta^2 +\frac{1}{2}\lambda v^2\xi^2\\
- \frac{1}{4}\lambda \left(v^4 + 4v^3\eta + 4v^2\eta^2 + 2v^2\eta^2 + 2v^2\xi^2 + 4v\eta^3 + 4v\eta\xi^2 + \eta^4 + 2\eta^2\xi^2 + \xi^4 \right)
\end{multline}

\begin{multline}
 \mathcal{L} = \frac{1}{2} (\partial^{\mu}\eta)(\partial_{\mu}\eta) + \frac{1}{2} (\partial^{\mu}\xi)(\partial_{\mu}\xi) + \frac{1}{2}\lambda v^4 + \lambda v^3\eta + \frac{1}{2}\lambda v^2\eta^2 +\frac{1}{2}\lambda v^2\xi^2\\
- \frac{1}{4}\lambda v^4 - \lambda v^3\eta - \lambda v^2\eta^2 - \frac{1}{2}\lambda v^2\eta^2  - \frac{1}{2}\lambda v^2\xi^2 - \lambda v\eta^3 - \lambda v\eta\xi^2 - \frac{1}{4}\lambda\eta^4 - \frac{1}{2}\lambda\eta^2\xi^2 + - \frac{1}{4}\lambda\xi^4 
\end{multline}

\begin{multline}
 \mathcal{L} = \frac{1}{2} (\partial^{\mu}\eta)(\partial_{\mu}\eta) + \frac{1}{2} (\partial^{\mu}\xi)(\partial_{\mu}\xi) - \lambda v^2\eta^2 - \lambda v\eta^3 - \lambda v\eta\xi^2 - \frac{1}{4}\lambda\eta^4 - \frac{1}{2}\lambda\eta^2\xi^2 + - \frac{1}{4}\lambda\xi^4 + \frac{1}{4}\lambda v^4
 \label{rs2}
\end{multline}

por lo que el lagrangiano describe un campo sin masa, $\xi$ denominado campo de Goldstone, y un campo masivo, $\eta$, de masa $m_{\mu}=\sqrt{2\lambda v}$. A excepción del último término que por ser constante no tiene relevancia física, el resto de términos correponden a la interacción y autointeracción de los campos $\eta$ y $\xi$, mostrados en la figura \ref{diag2}.

\begin{figure}[!h]
    \centering
    \begin{subfigure}[b]{0.3\textwidth}
        \begin{tikzpicture}
        \begin{feynman}
            \vertex[dot, thick, minimum size=1.0mm] (b) {};
            \vertex[left=of b](label){$\frac{1}{4}\lambda$};
            \vertex[above left=of b] (a) {$\eta$};
            \vertex[below left=of b] (c) {$\eta$};
            \vertex[below right=of b] (d) {$\eta$};
            \vertex[above right=of b] (e) {$\eta$};
            \diagram*{
                (a) -- [scalar] (b),
                (c) -- [scalar] (b),
                (d) -- [scalar] (b),
                (e) -- [scalar] (b),
            };
        \end{feynman}
        \end{tikzpicture}
    \end{subfigure}
    \begin{subfigure}[b]{0.3\textwidth}
        \begin{tikzpicture}
        \begin{feynman}
            \vertex[dot, thick, minimum size=1.0mm] (b) {};
            \vertex[left=of b](label){$\frac{1}{4}\lambda$};
            \vertex[above left=of b] (a) {$\xi$};
            \vertex[below left=of b] (c) {$\xi$};
            \vertex[below right=of b] (d) {$\xi$};
            \vertex[above right=of b] (e) {$\xi$};
            \diagram*{
                (a) -- [scalar] (b),
                (c) -- [scalar] (b),
                (d) -- [scalar] (b),
                (e) -- [scalar] (b),
            };
        \end{feynman}
        \end{tikzpicture}
    \end{subfigure}
    \begin{subfigure}[b]{0.3\textwidth}
        \begin{tikzpicture}
        \begin{feynman}
            \vertex[dot, thick, minimum size=1.0mm] (b) {};
            \vertex[left=of b](label){$\lambda v$};
            \vertex[above left=of b] (a) {$\eta$};
            \vertex[below left=of b] (c) {$\eta$};
            \vertex[right=of b] (d) {$\eta$};
            \diagram*{
                (a) -- [scalar] (b),
                (c) -- [scalar] (b),
                (d) -- [scalar] (b),
            };
        \end{feynman}
        \end{tikzpicture}
    \end{subfigure}
    \begin{subfigure}[b]{0.3\textwidth}
        \begin{tikzpicture}
        \begin{feynman}
            \vertex[dot, thick, minimum size=1.0mm] (b) {};
            \vertex[right=of b](label){$\lambda v$};
            \vertex[left=of b] (a) {$\eta$};
            \vertex[below right=of b] (c) {$\xi$};
            \vertex[above right=of b] (d) {$\xi$};
            \diagram*{
                (a) -- [scalar] (b),
                (c) -- [scalar] (b),
                (d) -- [scalar] (b),
            };
        \end{feynman}
        \end{tikzpicture}
    \end{subfigure}
    \begin{subfigure}[b]{0.3\textwidth}
        \begin{tikzpicture}
        \begin{feynman}
            \vertex[dot, thick, minimum size=1.0mm] (b) {};
            \vertex[right=of b](label){$\frac{1}{2}\lambda$};
            \vertex[above left=of b] (a) {$\eta$};
            \vertex[below left=of b] (c) {$\eta$};
            \vertex[below right=of b] (d) {$\xi$};
            \vertex[above right=of b] (e) {$\xi$};
            \diagram*{
                (a) -- [scalar] (b),
                (c) -- [scalar] (b),
                (d) -- [scalar] (b),
                (e) -- [scalar] (b),
            };
        \end{feynman}
        \end{tikzpicture}
    \end{subfigure}
    \caption{Interacción y autointeracción de los campos escales en el rompimiento de simetría de un campo complejo.}
    \label{diag2}
\end{figure}

\section{Rompimiento de simetría del potencial y simetría local de gauge U(1)}

El lagrangiano de la ecuación \ref{rs2} no es invariante ante transformaciones locales de fase $\phi \rightarrow \phi'=e^{ig\alpha (x)}\phi$, para que cumpla con ello en dicha expresión es necesario remplazar la derivada parcial por la derivada covariante, tal que 

$$ \partial_{\mu} \quad \longrightarrow \quad D_{\mu} = \partial_{\mu} + igB_{\mu} $$

en donde $B_{\mu}$ transforma

$$ B_{\mu} \quad \longrightarrow \quad B_{\mu}'= B_{\mu} - \partial_{\mu}\alpha(x) $$

remplazando en el lagrangiano

$$ \mathcal{L} = (D^{\mu}\phi)^*(D_{\mu}\phi) - V(\phi) $$

como se había visto con anterioridad, la simetría local de gauge implica la presencia de un campo de gauge, $B_{\mu}$, con una transoformación apropiada ante cambios de fase. Adicionando un término cinético correspondiente al campo de gauge

$$ \mathcal{L} = -\frac{1}{4}F^{\mu\nu}F_{\mu\nu} + (D^{\mu}\phi)^*(D_{\mu}\phi) - \mu^2\phi^2 - \lambda \phi^4 $$

donde $F^{\mu\nu} = \partial^{\mu}B^{\nu} - \partial^{\nu}B^{\mu}$. Descomponiendo el factor cinético del campo escalar complejo:

$$ (D^{\mu}\phi)^*(D_{\mu}\phi) = (\partial^{\mu} - igB^{\mu})\phi^* (\partial_{\mu} + igB_{\mu})\phi $$

$$ (D^{\mu}\phi)^*(D_{\mu}\phi) = (\partial^{\mu}\phi)^*(\partial_{\mu}\phi) + ig(\partial^{\mu}\phi)^*B_{\mu}\phi - igB^{\mu}\phi^*(\partial_{\mu}\phi) + g^2B^{\mu}B_{\mu}\phi^*\phi $$

Introduciendo este resultado

\begin{equation}
    \mathcal{L} = -\frac{1}{4}F^{\mu\nu}F_{\mu\nu} + (\partial^{\mu}\phi)^*(\partial_{\mu}\phi) - \mu^2\phi^2 - \lambda \phi^4 + ig(\partial^{\mu}\phi)^*B_{\mu}\phi - igB^{\mu}\phi^*(\partial_{\mu}\phi) + g^2B^{\mu}B_{\mu}\phi^*\phi
\end{equation}

teniendo $\phi = \frac{1}{\sqrt{2}}(\phi_1 + i\phi_2)$

\begin{multline}
    \mathcal{L} = -\frac{1}{4}F^{\mu\nu}F_{\mu\nu} + \frac{1}{2}(\partial^{\mu}\phi_1)(\partial_{\mu}\phi_1) + \frac{1}{2}(\partial^{\mu}\phi_2)(\partial_{\mu}\phi_2) -\frac{1}{2}\mu^2(\phi_1^2 + \phi_2^2)-\frac{1}{4}\lambda(\phi_1^2 + \phi_2^2)^2 +gB^{\mu}(\phi_1 (\partial_{\mu}\phi_2) - \phi_2(\partial_{\mu}\phi_1))\\
    +\frac{1}{2}g^2B^{\mu}B_{\mu}(\phi_1^2+\phi_2^2) 
\end{multline}

tomando el penúltimo término, sumando cero como $B^{\mu}(i\phi_2(\partial_{\mu}\phi_2) - i\phi_2(\partial_{\mu}\phi_2))$ y tomando $\phi_1+i\phi_2=v$

$$ gB^{\mu}(\phi_1 (\partial_{\mu}\phi_2) - \phi_2(\partial_{\mu}\phi_1)) = gB^{\mu}(\phi_1 (\partial_{\mu}\phi_2) + i\phi_2(\partial_{\mu}\phi_2) - i\phi_2(\partial_{\mu}\phi_2) - \phi_2(\partial_{\mu}\phi_1)) $$

$$ gB^{\mu}(\phi_1 (\partial_{\mu}\phi_2) - \phi_2(\partial_{\mu}\phi_1)) = gB^{\mu}v (\partial_{\mu}\phi_2) $$

\begin{equation}
    \mathcal{L} = -\frac{1}{4}F^{\mu\nu}F_{\mu\nu} + \frac{1}{2}(\partial^{\mu}\phi_1)(\partial_{\mu}\phi_1) + \frac{1}{2}(\partial^{\mu}\phi_2)(\partial_{\mu}\phi_2) -\frac{1}{2}\mu^2(\phi_1^2 + \phi_2^2)-\frac{1}{4}\lambda(\phi_1^2 + \phi_2^2)^2 + gB^{\mu}v (\partial_{\mu}\phi_2) +\frac{1}{2}g^2B^{\mu}B_{\mu}(\phi_1^2+\phi_2^2) 
\end{equation}

el campo escalar se toma ahora como perturbaciones del estado base $\phi(x) = v+\eta(x)+i\xi(x)$, tal que $\phi_1 = v+\eta(x)$ y $\phi_2=\xi(x)$

\begin{multline}
    \mathcal{L} = -\frac{1}{4}F^{\mu\nu}F_{\mu\nu} + \frac{1}{2}(\partial^{\mu}\eta)(\partial_{\mu}\eta) + \frac{1}{2}(\partial^{\mu}\xi)(\partial_{\mu}\xi)  - \lambda v^2\eta^2 - \lambda v\eta^3 - \lambda v\eta\xi^2 - \frac{1}{4}\lambda\eta^4 - \frac{1}{2}\lambda\eta^2\xi^2 + - \frac{1}{4}\lambda\xi^4 + \frac{1}{4}\lambda v^4 + gB^{\mu}v(\partial_{\mu}\xi)\\ 
    +\frac{1}{2}g^2B^{\mu}B_{\mu}(v^2 + \eta^2 + \xi^2 +2v\eta)
\end{multline}

\begin{multline}
    \mathcal{L} = -\frac{1}{4}F^{\mu\nu}F_{\mu\nu} + \frac{1}{2}(\partial^{\mu}\eta)(\partial_{\mu}\eta) + \frac{1}{2}(\partial^{\mu}\xi)(\partial_{\mu}\xi)  - \lambda v^2\eta^2 - \lambda v\eta^3 - \lambda v\eta\xi^2 - \frac{1}{4}\lambda\eta^4 - \frac{1}{2}\lambda\eta^2\xi^2 + - \frac{1}{4}\lambda\xi^4 + \frac{1}{4}\lambda v^4 + gvB^{\mu}(\partial_{\mu}\xi)\\ 
    +\frac{1}{2}g^2v^2B^{\mu}B_{\mu} + \frac{1}{2}g^2B^{\mu}B_{\mu}\eta^2 + \frac{1}{2}g^2B^{\mu}B_{\mu}\xi^2 +g^2vB^{\mu}B_{\mu}\eta
\end{multline}

nuevamente resulta un campo escalar masivo, $\eta$, un campo de Goldstone, $\xi$, y un campo de gauge que, consecuencia del rompimiento de simetría, presenta un factor de masa. Los demás factores corresponden a los términos de interacción y autointeracción de los campos, incluido el término de acoplamiento entre el campo de gauge y el campo de Goldstone, $gvB^{\mu}(\partial_{\mu}\xi)$. El campo de Goldstone puede ser removido del lagrangiano asignando una transformación adecuada para el campo de gauge. De esta forma

$$ \frac{1}{2}(\partial^{\mu}\xi)(\partial_{\mu}\xi) + gvB^{\mu}(\partial_{\mu}\xi) +\frac{1}{2}g^2v^2B^{\mu}B_{\mu} =\frac{1}{2}g^2v^2\left[B_{\mu} + \frac{1}{gv}(\partial_{\mu}\xi)\right]^2$$

obteniendo la transformación de gauge

$$ B_{\mu} \quad \longrightarrow \quad B_{\mu}' = B_{\mu} + \frac{1}{gv}(\partial_{\mu}\xi) $$

quedando el lagrangiano

\begin{multline}
    \mathcal{L} = -\frac{1}{4}F^{\mu\nu}F_{\mu\nu} + \frac{1}{2}(\partial^{\mu}\eta)(\partial_{\mu}\eta) - \lambda v^2\eta^2 - \lambda v\eta^3 - \lambda v\eta\xi^2 - \frac{1}{4}\lambda\eta^4 - \frac{1}{2}\lambda\eta^2\xi^2 + - \frac{1}{4}\lambda\xi^4 + \frac{1}{4}\lambda v^4\\ 
    +\frac{1}{2}g^2v^2B'^{\mu}B'_{\mu} + \frac{1}{2}g^2B^{\mu}B_{\mu}\eta^2 + \frac{1}{2}g^2B^{\mu}B_{\mu}\xi^2 +g^2vB^{\mu}B_{\mu}\eta
\end{multline}

por lo que la transformación adecuada del campo de gauge implica que la fase local de transformación viene dada por $\alpha(x) = -\xi(x)/gv$. Con dicha fase, la transformación toma la forma

$$ \phi \quad \longrightarrow \quad \phi' = e^{-g\xi(x)/gv}\phi = e^{-\xi(x)/v}\phi $$

en el rompimiento de simetría la función de onda, tomada como perturbaciones del estado base $\phi=\frac{1}{\sqrt{2}}(v+\eta(x)+i\xi(x))$, puede ser expresado a primer orden como 

$$ \phi \approx \frac{1}{\sqrt{2}}(v+\eta(x))e^{i\xi(x)/v} $$

de esta forma, la transformación de fase es expresada como

$$ \phi \quad \longrightarrow \quad \phi' = e^{-\xi(x)/v}\phi \approx\frac{1}{\sqrt{2}} e^{-\xi(x)/v}  (v+\eta(x))e^{i\xi(x)/v} = \frac{1}{\sqrt{2}}(v + \eta(x))  $$

por lo que la transformación de fase, a primer orden, implica una función de onda completamente real con la que, además, el campo de Goldstone no aparece; esta condición es denominada gauge unitario. Remplazando en el lagrangiano

$$ \mathcal{L} = (D_{\mu}\phi)^*(D^{\mu}\phi) -\frac{1}{4}F^{\mu\nu}F_{\mu\nu} - \mu^2\phi^2 - \lambda \phi^4 $$

$$ \mathcal{L} = \frac{1}{2}(\partial^{\mu} - igB^{\mu})(v+\eta)(\partial_{\mu} + igB_{\mu})(v+\eta) - \frac{1}{4}F^{\mu\nu}F_{\mu\nu} -\frac{1}{2}\mu^2 (v+\eta)^2 - \frac{1}{4}\lambda (v+\eta)^4  $$

$$ \mathcal{L} = \frac{1}{2}(\partial^{\mu}\eta)(\partial_{\mu}\eta) + \frac{1}{2}g^2B^{\mu}B_{\mu}(v+\eta)^2 - \frac{1}{4}F^{\mu\nu}F_{\mu\nu} -\lambda v^2 \eta^2 -\lambda v \eta^3 -\frac{1}{4}\lambda \eta^4 + \frac{1}{4}\lambda v^4 $$

$$ \mathcal{L} = \frac{1}{2}(\partial^{\mu}\eta)(\partial_{\mu}\eta) -\lambda v^2 \eta^2 - \frac{1}{4}F^{\mu\nu}F_{\mu\nu} + \frac{1}{2}g^2v^2 B^{\mu}B_{\mu} + g^2vB^{\mu}B_{\mu}\eta + \frac{1}{2}g^2B^{\mu}B_{\mu}\eta^2 -\lambda v \eta^3 -\frac{1}{4}\lambda \eta^4 + \frac{1}{4}\lambda v^4 $$

renombrando el campo $\eta(x)\equiv h(x)$, lo cual indica el manejo del gauge unitario. Reescribiendo:

\begin{equation}
    \mathcal{L} = \frac{1}{2}(\partial^{\mu}h)(\partial_{\mu}h) -\lambda v^2 h^2 - \frac{1}{4}F^{\mu\nu}F_{\mu\nu} + \frac{1}{2}g^2v^2 B^{\mu}B_{\mu} + g^2vB^{\mu}B_{\mu}h + \frac{1}{2}g^2B^{\mu}B_{\mu}h^2 -\lambda v h^3 -\frac{1}{4}\lambda h^4 + \frac{1}{4}\lambda v^4
\end{equation}

Los dos primeros términos corresponden al componente cinético y de masa del campo $h(x)$, respectivamente, mientras que los dos siguientes corresponden al componente cinético y de masa del campo de gauge $B$, respectivamente. Tomando los términos de masa se tiene que 

$$ m_B = gv \quad , \quad m_h = \sqrt{2\lambda}v $$

por lo que los valores de las masas dependen explícitamente del valor esperado del campo $\phi$ en el vacío, $v$. A excepción del último término, que es constante, los términos restantes corresponden a la interacción del campo $h$ con el campo $B$ y la autointeracción del campo $h$, como es mostrado en la figura \ref{fig:enter-label}.

\begin{figure}[!h]
    \centering
    \begin{subfigure}[b]{0.2\linewidth}
        \begin{tikzpicture}
        \begin{feynman}
            \vertex[dot, thick, minimum size=1.0mm] (b) {};
            \vertex[right=1.7em of b](label){$g^2v$};
            \vertex[left=of b] (a) {$h$};
            \vertex[below right=of b] (d) {$B$};
            \vertex[above right=of b] (e) {$B$};
            \diagram*{
                (a) -- [scalar] (b),
                (d) -- [photon] (b),
                (e) -- [photon] (b),
            };
        \end{feynman}
        \end{tikzpicture}
    \end{subfigure}
    \begin{subfigure}[b]{0.2\linewidth}
        \begin{tikzpicture}
        \begin{feynman}
            \vertex[dot, thick, minimum size=1.0mm] (b) {};
            \vertex[right=1.7em of b](label){$\frac{1}{2}g^2$};
            \vertex[below left=of b] (a) {$h$};
            \vertex[above left=of b] (c) {$h$};
            \vertex[below right=of b] (d) {$B$};
            \vertex[above right=of b] (e) {$B$};
            \diagram*{
                (a) -- [scalar] (b),
                (c) -- [scalar] (b),
                (d) -- [photon] (b),
                (e) -- [photon] (b),
            };
        \end{feynman}
        \end{tikzpicture}
    \end{subfigure}
    \begin{subfigure}[b]{0.2\linewidth}
        \begin{tikzpicture}
        \begin{feynman}
            \vertex[dot, thick, minimum size=1.0mm] (b) {};
            \vertex[right=1.5em of b](label){$\lambda v$};
            \vertex[left=of b] (a) {$h$};
            \vertex[below right=of b] (d) {$h$};
            \vertex[above right=of b] (e) {$h$};
            \diagram*{
                (a) -- [scalar] (b),
                (d) -- [scalar] (b),
                (e) -- [scalar] (b),
            };
        \end{feynman}
        \end{tikzpicture}
    \end{subfigure}
    \begin{subfigure}[b]{0.2\linewidth}
        \begin{tikzpicture}
        \begin{feynman}
            \vertex[dot, thick, minimum size=1.0mm] (b) {};
            \vertex[right=1.4em of b](label){$\frac{1}{4}\lambda$};
            \vertex[below left=of b] (a) {$h$};
            \vertex[above left=of b] (c) {$h$};
            \vertex[below right=of b] (d) {$h$};
            \vertex[above right=of b] (e) {$h$};
            \diagram*{
                (a) -- [scalar] (b),
                (c) -- [scalar] (b),
                (d) -- [scalar] (b),
                (e) -- [scalar] (b),
            };
        \end{feynman}
        \end{tikzpicture}
    \end{subfigure}
    \caption{Diagramas de interacción del campo $h$ con el campo $B$ y autointeracción del campo $h$.}
    \label{fig:enter-label}
\end{figure}
