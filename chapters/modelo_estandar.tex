\chapter{Modelo estándar}

\section{Simetría quiral}

La función de onda de un fermión puede presentar una descomposición en grados de libertad independientes dados por las proyecciones izquierda y derecha tal que

$$ \psi = \left[\frac{1}{2}(I+\gamma^5) + \frac{1}{2}(I-\gamma^5) \right]\psi = P_R\psi + P_L\psi = \psi_R + \psi_L $$

donde $P_R=\frac{1}{2}(I+\gamma^5)$ y $P_L=\frac{1}{2}(I-\gamma^5)$ son las matrices de proyección derecha e izquierda, respectivamente, que cumplen la condición:

$$ P_R^2=P_R \quad , \quad P_L^2=P_L \quad , \quad P_RP_L=P_LP_R = 0 $$

Dicha consideración es de particular relevancia pues, según se ha observado(\cite{wu1957experimental}), una fuerza fundamental de la naturaleza (la fuerza débil) actúa de forma asimétrica para las distintas proyecciones: la fuerza débil actúa solamente sobre el componente izquierdo de la función de onda. Para ello es necesario considerar un número cuántico adicional para la función de onda que determina si la partícula interactúa mediante fuerza débil, denominado isospin débil, cuya componente de interés, $T_3$, toma los valores 0, $\frac{1}{2}$ y $-\frac{1}{2}$. Un fermión interactúa mediante fuerza débil si el valor absoluto de su isospin débil es $\frac{1}{2}$ y no interactúa mediante fuerza débil si el valor absoluto de su isospin débil es 0. Dado que la interacción débil es producto de una simetría de gauge del grupo $SU(2)$, para mostrar explícitamente el hecho de que únicamente actúa sobre las funciones de quiralidad izquierda se denota $SU(2)_L$. Teniendo para los componentes del lagrangiano de un fermión(\cite{cheng2000gauge}):

$$ \bar{\psi}\psi = \bar{\psi} \left[\frac{1}{2}(I+\gamma^5) + \frac{1}{2}(I-\gamma^5) \right] \psi = \frac{1}{2}\bar{\psi}(I+\gamma^5)\psi + \frac{1}{2}\bar{\psi}(I-\gamma^5)\psi = \bar{\psi}_L\psi_R + \bar{\psi}_R\psi_L $$

y 

$$ \bar{\psi}i\gamma^{\mu}\partial_{\mu}\psi = \bar{\psi}_Li\gamma^{\mu}\partial_{\mu}\psi_L + \bar{\psi}_Ri\gamma^{\mu}\partial_{\mu}\psi_R $$

\section{Rompimiento espontáneo de simetría con simetría gauge SU(2)$_L\times$U(1)$_Y$}

El modelo de Higgs considera dos campos escalares en un doblete de isospin débil tal que

\begin{align}
    \phi &= \begin{pmatrix}
           \phi^+ \\
           \phi^0 \\
         \end{pmatrix}
         = \frac{1}{\sqrt{2}} \begin{pmatrix}
            \phi_1+i\phi_2\\
            \phi_3+i\phi_4\\
         \end{pmatrix}
  \end{align}

con $\phi^0$ un campo neutro y $\phi^+$ un campo cargado tal que $(\phi^+)^* = \phi^-$, difiriendo un campo del otro en una unidad de carga. El lagrangiano del doblete viene dado por

$$ \mathcal{L} = (\partial^{\mu}\phi)^{\dagger} (\partial_{\mu}\phi) - \mu^2 \phi^{\dagger}\phi - \lambda(\phi^{\dagger}\phi)^2 $$

por lo que, de igual manera que para los casos anteriores, se tiene que para $\mu^2<0$ el potencial tiene un conjunto infinito de valores mínimos que cumplen la condición

$$ \phi_1^2 + \phi_2^2 + \phi_3^2 + \phi_4^2 = v^2 $$

con $v^2 =-\mu^2/\lambda$. El bosón mediador de la fuerza electromagnética debe permanecer sin masa, por lo que el valor esperado en el vacío debe tener un valor diferente de cero solo para $\phi^0$

\begin{align}
    \bra{0}\phi\ket{0} = \frac{1}{\sqrt{2}}\begin{pmatrix}
        0\\
        v\\
        \end{pmatrix}
\end{align}

tomando el campo como perturbaciones del vacío 

\begin{align}
    \phi = \frac{1}{\sqrt{2}}\begin{pmatrix}
        \phi_1(x)+i\phi_2(x)\\
        v+\eta(x)+i\phi_4(x)\\
        \end{pmatrix}
\end{align}

escribiendo el doblete en su forma de gauge unitario

\begin{align}
    \phi = \frac{1}{\sqrt{2}}\begin{pmatrix}
        0\\
        v+h(x)\\
        \end{pmatrix}
\end{align}

Los términos de masa de los bosones generados pueden ser identificados modificando el lagrangiano tal que sea simétrico ante la simetría local $SU(2)_L\times U(1)_Y$. Esto se logra remplazando la derivada parcial por la derivada covariante adecuada: 

$$ \partial_{\mu} \quad \longrightarrow \quad D_{\mu} = \partial_{\mu} + ig \textbf{T}\cdot \textbf{W}_{\mu} +ig'\frac{Y}{2}B_{\mu} $$

donde $\textbf{T} = \frac{1}{2}\boldsymbol\sigma$, los generadores de la simetría $SU(2)$, y $Y$ es la hipercarga débil definida como $Y = 2(Q - T_3)$, con $Q$ la carga eléctrica. Tomando el valor del isospin débil del primer campo como $1/2$ y el del segundo como $-1/2$, se tiene que $Y=1$. Por lo que la derivada covariante actuando sobre la función de onda resulta

$$ D_{\mu}\phi = \frac{1}{2}\left[2\partial_{\mu} + \left( ig\boldsymbol\sigma \cdot \textbf{W}_{\mu} + ig'B_{\mu} \right) \right]\phi $$

lo que representa una matriz $2\text{x}2$, $D_{\mu}$, multiplicando el doblete de isospin. En el gauge unitario:

$$ D_{\mu}\phi  = \frac{1}{2\sqrt{2}}\begin{pmatrix}
    2\partial_{\mu} + igW_{\mu}^{(3)}+ig'B_{\mu} & ig(W_{\mu}^{(1)} -iW_{\mu}^{(2)}) \\
    ig(W_{\mu}^{(1)} +iW_{\mu}^{(2)}) & 2\partial_{\mu} - igW_{\mu}^{(3)}+ig'B_{\mu}\\
\end{pmatrix}
    \begin{pmatrix}
        0 \\
        v+h\\
    \end{pmatrix}$$

$$ D_{\mu}\phi = \frac{1}{2\sqrt{2}}\begin{pmatrix}
    ig(W_{\mu}^{(1)} -iW_{\mu}^{(2)})(v+h)\\
    (2\partial_{\mu} - igW_{\mu}^{(3)}+ig'B_{\mu})(v+h) \\
\end{pmatrix} $$

de esta forma

\begin{multline}
    (D_{\mu}\phi)^{\dagger}(D^{\mu}\phi) = \frac{1}{8}\times \\
    \times \begin{pmatrix}
        -ig(W_{\mu}^{(1)} +iW_{\mu}^{(2)})(v+h), &
        2\partial_{\mu}h + i(gW_{\mu}^{(3)}-g'B_{\mu})(v+h)
    \end{pmatrix} \begin{pmatrix}
        ig(W^{(1)\mu} -iW^{(2)\mu})(v+h)\\
        2\partial^{\mu}h - i(gW^{\mu(3)}-g'B^{\mu})(v+h) \\
    \end{pmatrix} 
\end{multline}

\begin{multline}
(D_{\mu}\phi)^{\dagger}(D^{\mu}\phi) = \frac{1}{2}(\partial_{\mu}h)(\partial^{\mu}h) +\frac{1}{8}g^2(W_{\mu}^{(1)} +iW_{\mu}^{(2)})(W^{(1)\mu} -iW^{(2)\mu})(v+h)^2 +\\
\frac{1}{8}(gW_{\mu}^{(3)}-g'B_{\mu})(gW^{\mu(3)}-g'B^{\mu})(v+h)^2
\end{multline}

definiendo

$$ W^{(\pm)}_{\mu} = \frac{1}{\sqrt{2}}(W^{(1)}_{\mu}\mp i W^{(2)}_{\mu}) $$

se tiene

\begin{multline}
    (D_{\mu}\phi)^{\dagger}(D^{\mu}\phi) = \frac{1}{2}(\partial_{\mu}h)(\partial^{\mu}h) +\frac{1}{4}g^2W_{\mu}^{(+)}W^{(-)\mu}(v^2+2vh+h^2) +\dots \\
     \frac{1}{8}(gW_{\mu}^{(3)}-g'B_{\mu})(gW^{\mu(3)}-g'B^{\mu})(v+h)^2
\end{multline}

\begin{multline}
    (D_{\mu}\phi)^{\dagger}(D^{\mu}\phi) = \frac{1}{2}(\partial_{\mu}h)(\partial^{\mu}h) +\frac{1}{8}v^2g^2W_{\mu}^{(+)}W^{(-)\mu} + \frac{1}{8}(gW_{\mu}^{(3)}-g'B_{\mu})(gW^{\mu(3)}-g'B^{\mu})(v+h)^2 \\
    +\frac{1}{4}vg^2W_{\mu}^{(+)}W^{(-)\mu}h +\frac{1}{8}g^2W_{\mu}^{(+)}W^{(-)\mu}h^2
\end{multline}

\begin{multline}
    (D_{\mu}\phi)^{\dagger}(D^{\mu}\phi) = \frac{1}{2}(\partial_{\mu}h)(\partial^{\mu}h) +\frac{1}{8}v^2g^2(W_{\mu}^{(1)}W^{(1)\mu} + W_{\mu}^{(2)}W^{(2)\mu}) + \frac{1}{8}v^2(gW_{\mu}^{(3)}-g'B_{\mu})(gW^{\mu(3)}-g'B^{\mu}) \\
    +\frac{1}{4}vg^2(W_{\mu}^{(1)}W^{(1)\mu} + W_{\mu}^{(2)}W^{(2)\mu})h +\frac{1}{8}g^2(W_{\mu}^{(1)}W^{(1)\mu} + W_{\mu}^{(2)}W^{(2)\mu})h^2 + \frac{1}{4}v(gW_{\mu}^{(3)}-g'B_{\mu})(gW^{\mu(3)}-g'B^{\mu})h\\ + \frac{1}{8}(gW_{\mu}^{(3)}-g'B_{\mu})(gW^{\mu(3)}-g'B^{\mu})h^2
    \label{cc}
\end{multline}

nuevamente los términos cuadráticos de los campos determinan el valor de las masas de los bosones mediadores. Para los campos $W^{(1)}$ y $W^{(2)}$ se tiene que 

$$ \frac{1}{8}v^2g^2W_{\mu}^{(1)}W^{(1)\mu} + \frac{1}{8}v^2g^2 W_{\mu}^{(2)}W^{(2)\mu} = \frac{1}{2}m_WW_{\mu}^{(1)}W^{(1)\mu} + \frac{1}{2}m_WW_{\mu}^{(2)}W^{(2)\mu} $$

con lo que 

$$ m_W = \frac{1}{2}vg $$

por lo que la masa de los bosones W depende de la constante de acoplamiento del grupo de simetría $SU(2)_L$ y el valor esperado del campo de Higgs en el vacío. El término cuadrático de los campos $W^{(3)}$ y $B$ se puede expresar como

$$ \frac{1}{8}v^2(gW_{\mu}^{(3)}-g'B_{\mu})(gW^{\mu(3)}-g'B^{\mu}) = \frac{v^2}{8}\begin{pmatrix}
    W_{\mu}^{(3)} & B_{\mu}
\end{pmatrix} \begin{pmatrix}
    g^2 & -gg' \\
    -gg' & g'^2\\
\end{pmatrix} \begin{pmatrix}
    W^{(3)\mu}\\
    B^{\mu}\\
\end{pmatrix}$$

$$ \frac{1}{8}v^2(gW_{\mu}^{(3)}-g'B_{\mu})(gW^{\mu(3)}-g'B^{\mu}) = \frac{v^2}{8}\begin{pmatrix}
    W_{\mu}^{(3)} & B_{\mu}
\end{pmatrix}\textbf{M} \begin{pmatrix}
    W^{(3)\mu}\\
    B^{\mu}\\
\end{pmatrix} $$

donde \textbf{M} es la matriz de masa en su forma no diagonal, donde los términos fuera de la diagonal permiten el acoplamiento de los dos campos. Los campos físicos de los bosones se propagan como los autoestados del hamiltoniano de la partícula libre, los cuales corresponden a la base en la que la matriz \textbf{M} es diagonal. La masa de los bosones de gauge vienen dadas por los autovalores de la matriz \textbf{M}, los cuales son soluciones de la ecuación característica $det(\textbf{M} - \lambda I)=0$:

$$ (g^2 - \lambda)(g'^2 -\lambda) - g^2g'^2 = 0 $$

$$ g^2g'^2 -(g^2+g'^2)\lambda + \lambda^2  - g^2g'^2 = 0  $$

$$ \lambda(\lambda - (g^2+g'^2)) = 0 $$

$$ \lambda = 0 \quad , \quad \lambda = g^2+g'^2 $$

por lo que el producto matricial anterior puede ser expresado en la base matricial como

$$ \frac{v^2}{8}\begin{pmatrix}
    A_{\mu} & Z_{\mu}
\end{pmatrix}\begin{pmatrix}
    0 & 0 \\
    0 & g^2+g'^2\\
\end{pmatrix}\begin{pmatrix}
    A^{\mu} \\
    Z^{\mu}\\
\end{pmatrix} $$

donde $A_{\mu}$ y $Z_{\mu}$ son los campos físicos correspondientes a los autovalores de la matriz \textbf{M}. Como en la base diagonal los elementos de la diagonal corresponden a las masas de los campos, se puede expresar como 

$$ \frac{1}{2}\begin{pmatrix}
    A_{\mu} & Z_{\mu}
\end{pmatrix}\begin{pmatrix}
    m_A^2 & 0 \\
    0 & m_Z^2\\
\end{pmatrix}\begin{pmatrix}
    A^{\mu} \\
    Z^{\mu}\\
\end{pmatrix} $$

donde 

$$ m_A = 0 \quad , \quad m_Z = \frac{1}{2}v\sqrt{g^2+g'^2} $$

por lo que en la representación en la base diagonal resulta en un boson $A$ sin masa, el cual puede ser interpretado como el fotón mediador de la fuerza electromagnética, y un boson $Z$ con masa $m_Z = \frac{1}{2}v\sqrt{g^2+g'^2}$, que puede ser interpretado como el boson $Z$ mediador de la fuerza débil. Los campos físicos, obtenidos de la normalización de los autovalores de la matriz de masa, quedan

$$ A_{\mu} = \frac{g'W_{\mu}^{(3)} + gB_{\mu}}{\sqrt{g^2+g'^2}} $$
$$ Z_{\mu} = \frac{gW_{\mu}^{(3)} - g'B_{\mu}}{\sqrt{g^2+g'^2}} $$

El boson físico $W$ puede ser expresado como una combinación lineal

$$ W^{\pm} = \frac{1}{\sqrt{2}} (W^{(1)} \mp iW^{(2)}) $$

de la ecuación \ref{cc} se tiene la interacción de los campos $W^{\pm}$ con el campo de Higgs. Para el término $\frac{1}{2}vg^2W_{\mu}^-W^{+\mu}h$, el factor de acoplamiento del campo de Higgs con los bosones $W^+$ y $W^-$ con el boson de Higgs 

$$ \frac{1}{2}vg^2 = gm_W $$

depende de la masa de los bosones $W$ (que es igual para ambos). De igual forma, el factor de acoplamiento del campo de Higgs con el boson $Z$ se obtiene del factor

$$ \frac{1}{4}v(g^2+g'^2)Z_{\mu}Z^{\mu}h=\frac{m_Z^2}{v} Z_{\mu}Z^{\mu}h $$

multiplicando por $\frac{2g}{2g}$

$$ \frac{1}{4}v(g^2+g'^2)Z_{\mu}Z^{\mu}h=\frac{2g}{2g} \frac{m_Z^2}{v} Z_{\mu}Z^{\mu}h = \frac{1}{2}g\frac{m_Z^2}{m_W} Z_{\mu}Z^{\mu}h 
 = \frac{1}{2} g_Zm_Z Z_{\mu}Z^{\mu}h $$

con $g_Z = g\frac{m_Z}{m_W}$, el factor de acoplamiento viene dado por $\frac{1}{2}g_Zm_Z$. Los diagramas de estas interacciones son mostrados en la figura \ref{hwwz}.

\begin{figure}[!h]
    \centering
    \begin{subfigure}[b]{0.3\linewidth}
        \begin{tikzpicture}
            \begin{feynman}
                \vertex[dot, thick, minimum size=1.0mm] (b) {};
            \vertex[right=2.4em of b](label){$gm_W$};
            \vertex[left=of b] (a) {$h$};
            \vertex[below right=of b] (d) {$W^+$};
            \vertex[above right=of b] (e) {$W^-$};
            \diagram*{
                (a) -- [scalar] (b),
                (d) -- [photon] (b),
                (e) -- [photon] (b),
            };
            \end{feynman}
        \end{tikzpicture}
    \end{subfigure}
    \begin{subfigure}[b]{0.3\linewidth}
        \begin{tikzpicture}
            \begin{feynman}
                \vertex[dot, thick, minimum size=1.0mm] (b) {};
            \vertex[right=3.4em of b](label){$\frac{1}{2}g_Zm_Z$};
            \vertex[left=of b] (a) {$h$};
            \vertex[below right=of b] (d) {$Z$};
            \vertex[above right=of b] (e) {$Z$};
            \diagram*{
                (a) -- [scalar] (b),
                (d) -- [photon] (b),
                (e) -- [photon] (b),
            };
            \end{feynman}
        \end{tikzpicture}
    \end{subfigure}
    \caption{Diagramas de interacción del bosón de Higgs con los bosones $W^{\pm}$ y con el bosón $Z$.}
    \label{hwwz}
\end{figure}

\begin{multline}
    (D_{\mu}\phi)^{\dagger}(D^{\mu}\phi) = \frac{1}{2}(\partial_{\mu}h)(\partial^{\mu}h) +\frac{1}{2}m_WW_{\mu}^+ W^{+\mu} + \frac{1}{2}m_WW_{\mu}^- W^{-\mu} + \frac{1}{2}m_Z^2 Z_{\mu}Z^{\mu} +gm_WW_{\mu}^-W^{+\mu}h\\ 
    +\frac{1}{4}g^2W_{\mu}^-W^{+\mu}h^2 + \frac{1}{2}g_Zm_Z Z_{\mu}Z^{\mu}h + \frac{1}{8}g_Z^2Z_{\mu}Z^{\mu}h^2
\end{multline}

añadiendo los factores del potencial 

\begin{multline}
    \mathcal{L} = \frac{1}{2}(\partial_{\mu}h)(\partial^{\mu}h) +\frac{1}{2}m_WW_{\mu}^+ W^{+\mu} + \frac{1}{2}m_WW_{\mu}^- W^{-\mu} + \frac{1}{2}m_Z^2 Z_{\mu}Z^{\mu} +gm_WW_{\mu}^-W^{+\mu}h +\frac{1}{4}g^2W_{\mu}^-W^{+\mu}h^2\\
    + \frac{1}{2}g_Zm_Z Z_{\mu}Z^{\mu}h + \frac{1}{8}g_Z^2Z_{\mu}Z^{\mu}h^2 -\lambda v^2h^2-\lambda vh^3-\frac{1}{4}\lambda h^4 - \frac{1}{4}\lambda v^4
\end{multline}

\section{Lagrangiano fermiónico}

Los factores correspondientes a la energía cinética de los fermiones viene dado por el primer término de la ecuación de Dirac con la derivada covariante para el caso de la simetría de gauge $SU(2)_L\times U(1)_Y$. En este caso se tiene

$$ \mathcal{L} = \bar{\psi}(i\gamma^{\mu}D_{\mu})\psi $$

en donde la derivada covariante es 

$$ D_{\mu} = \partial_{\mu} + ig\textbf{T}\cdot \textbf{W}_{\mu} + ig'\frac{Y}{2}B_{\mu} $$

para el caso de un fermión de isospin débil $T_3 = \pm 1/2$ y $Y=1$, la derivada covariante actúa sobre la función de onda de la forma

$$ D_{\mu}\psi = \left[\partial_{\mu} + ig\frac{\sigma}{2}\cdot \textbf{W}_{\mu} + ig'\frac{1}{2}B_{\mu} \right]\psi $$

$$ D_{\mu}\psi = \left[\partial_{\mu} + ig\frac{\sigma}{2}\cdot \textbf{W}_{\mu} + ig'\frac{1}{2}B_{\mu} \right]\begin{pmatrix}
    \psi_L \\
    \psi_R\\\end{pmatrix}$$

el cual puede ser representado como(\cite{peskin2018introduction})

$$ D_{\mu} = \partial_{\mu} -i\frac{g}{\sqrt{2}}(W_{\mu}^{+}T^{+} + W_{\mu}^{-}T^{-}) - i\frac{g}{\text{cos}\theta_W}Z_{\mu}(T^3 - \text{sen}^2 \theta_W Q) - ieA_{\mu}Q $$

con $Q=T^3+Y$ y $e=g\text{sen}\theta_W$. Por lo que se tiene en general para los fermiones y teniendo en cuenta que no existe quiralidad derecha para antineutrinos

\begin{equation}
    \mathcal{L} = 
\overline{l}_L (i \slashed{\partial}) l_L + 
\overline{e}_R (i \slashed{\partial}) e_R + 
\overline{Q}_L (i \slashed{\partial}) Q_L + 
\overline{u}_R (i \slashed{\partial}) u_R + 
\overline{d}_R (i \slashed{\partial}) d_R \\
+ g \big(W_\mu^+ J_W^{\mu+} + W_\mu^- J_W^{\mu-} + Z_\mu^0 J_Z^\mu\big) + e A_\mu J_\text{EM}^\mu
\end{equation}

con 

\begin{align}
    J_W^{\mu+} &= \frac{1}{\sqrt{2}} \big( \overline{\nu}_L \gamma^\mu e_L + \overline{u}_L \gamma^\mu d_L \big); \\
    J_W^{\mu-} &= \frac{1}{\sqrt{2}} \big( \overline{e}_L \gamma^\mu \nu_L + \overline{d}_L \gamma^\mu u_L \big); \\
    J_Z^\mu &= \frac{1}{\cos \theta_w} \Big[
        \overline{\nu}_L \gamma^\mu \big( \frac{1}{2} \big) \nu_L + 
        \overline{e}_L \gamma^\mu \big( -\frac{1}{2} + \sin^2 \theta_w \big) e_L + 
        \overline{e}_R \gamma^\mu \big( \sin^2 \theta_w \big) e_R \notag \\
        &\quad + \overline{u}_L \gamma^\mu \big( \frac{1}{2} - \frac{2}{3} \sin^2 \theta_w \big) u_L + 
        \overline{u}_R \gamma^\mu \big( -\frac{2}{3} \sin^2 \theta_w \big) u_R \notag \\
        &\quad + \overline{d}_L \gamma^\mu \big( -\frac{1}{2} + \frac{1}{3} \sin^2 \theta_w \big) d_L + 
        \overline{d}_R \gamma^\mu \big( \frac{1}{3} \sin^2 \theta_w \big) d_R 
    \Big]; \\
    J_\text{EM}^\mu &= \overline{e} \gamma^\mu (-1) e + 
    \overline{u} \gamma^\mu \big( +\frac{2}{3} \big) u + 
    \overline{d} \gamma^\mu \big( -\frac{1}{3} \big) d.
\end{align}