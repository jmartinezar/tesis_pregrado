\chapter{Ecuación de Dirac}

\section{Derivación}

Una descripción relativista de la mecánica cuántica requiere una ecuación de movimiento que sea invariante de Lorentz. Un primer atisbo para dicha construcción proviene de la ecuación de Klein-Gordon que se deriva de la relación relativista de energía-momento 

$$ E^2 = \textbf{p}^2 +m^2 $$

considerando ambos lados del igual en su forma de operador actuando sobre un estado físico $\psi$

$$ \hat{E}^2\psi = \hat{\textbf{p}}^2\psi + m^2\psi $$

usando la representación de los operadores energía y momento en forma de operador diferencial

$$ E = i\frac{\partial}{\partial t} \quad , \quad \textbf{p} = -i\boldsymbol\nabla $$

se obtiene

\begin{equation}
    \frac{\partial^2 \psi}{\partial t^2} = \boldsymbol\nabla^2\psi -m^2\psi
    \label{kg}
\end{equation}

igualando a cero

$$ (\partial^{\mu}\partial_{\mu} + m^2)\psi = 0 $$

haciendo $\psi^*\times (\ref{kg}) - \psi \times (\ref{kg})^*$


$$ \psi^*\frac{\partial^2 \psi}{\partial^2 t} - \psi\frac{\partial^2 \psi^*}{\partial^2 t} = \psi^*(\boldsymbol\nabla^2\psi -m^2\psi) - \psi (\boldsymbol\nabla^2\psi^* -m^2\psi^*) $$

$$ \frac{\partial}{\partial t}\left( \psi^*\frac{\partial \psi}{\partial t} - \psi\frac{\partial \psi^*}{\partial t} \right) = \boldsymbol\nabla \cdot (\psi^*\boldsymbol\nabla \psi - \psi\boldsymbol\nabla \psi^*) $$

como se tiene la ecuación de continuidad de la forma

$$ \boldsymbol\nabla \textbf{j} + \frac{\partial \rho}{\partial t} = 0 $$

por comparación se obtiene

$$ \textbf{j} = -i(\psi^*\boldsymbol\nabla \psi - \psi\boldsymbol\nabla \psi^*) $$

$$ \rho = i\left( \psi^*\frac{\partial \psi}{\partial t} - \psi\frac{\partial \psi^*}{\partial t} \right) $$

donde el factor $i$ es puesto para asegurar una corriente real. Tomando la solución de onda plana

$$ \psi(\textbf{x},t) = Ne^{i(\textbf{p}\cdot\textbf{x} - Et)} $$

se tiene para la corriente

$$ \textbf{j} = -i(iNe^{-i(\textbf{p}\cdot\textbf{x} - Et)}\textbf{p} Ne^{i(\textbf{p}\cdot\textbf{x} - Et)}+iNe^{i(\textbf{p}\cdot\textbf{x} - Et)}\textbf{p}Ne^{-i(\textbf{p}\cdot\textbf{x} - Et)})$$

$$ \textbf{j} = 2N^2\textbf{p} $$

y para la densidad de probabilidad

$$ \rho = i(-iENe^{-i(\textbf{p}\cdot\textbf{x} - Et)}Ne^{i(\textbf{p}\cdot\textbf{x} - Et)} -iENe^{i(\textbf{p}\cdot\textbf{x} - Et)}Ne^{-i(\textbf{p}\cdot\textbf{x} - Et)})$$

$$ \rho = 2N^2E $$

por lo que para energías negativas la probabilidad toma valores ajenos a la interpretación física. Por lo que dicha solución no cumple los requerimientos físicos que debe cumplir una ecuación general de movimiento. La idea de Dirac fue tomar una forma modificada de la relación de energía-momento que asocie la energía linealmente con el momento y con la masa, tal que

$$ \hat{E}\psi = (\boldsymbol\alpha \cdot \hat{\textbf{p}} + \beta m)\psi $$

$$ i\frac{\partial \psi}{\partial t} = \left( -i\alpha_x\partial_x -i\alpha_y\partial_y -i\alpha_x\partial_x +\beta m \right)\psi $$

tomando el cuadrado del operador 

$$ -\frac{\partial^2 \psi}{\partial t^2} = \left( i\alpha_x\partial_x +i\alpha_y\partial_y +i\alpha_x\partial_x -\beta m \right)\left( i\alpha_x\partial_x +i\alpha_y\partial_y +i\alpha_x\partial_x -\beta m \right)\psi $$

desplegando

\begin{multline}
    \frac{\partial^2 \psi}{\partial t^2} = \alpha_x^2 \partial_x^2\psi + \alpha_y^2 \partial_y^2\psi + \alpha_z^2 \partial_z^2\psi - \beta^2m^2 \\
    +(\alpha_x \alpha_y + \alpha_y \alpha_x)\frac{\partial^2}{\partial x \partial y}+(\alpha_y \alpha_z + \alpha_z \alpha_y)\frac{\partial^2}{\partial y \partial z}+(\alpha_z \alpha_x + \alpha_x \alpha_z)\frac{\partial^2}{\partial z \partial x}\\
    +i(\alpha_x \beta + \beta \alpha_x)m\frac{\partial \psi}{\partial x} +i(\alpha_y \beta + \beta \alpha_y)m\frac{\partial \psi}{\partial y}+i(\alpha_z \beta + \beta \alpha_z)m\frac{\partial \psi}{\partial z}
\end{multline}

como esta relación debe satisfacer la ecuación de Klein-Gordon, se tiene

$$ \alpha_x^2 = \alpha_y^2 + \alpha_z^2 = \beta^2 =  I $$

$$ \alpha_i\beta + \beta\alpha_i = 0 $$

$$ \alpha_i\alpha_j + \alpha_j\alpha_i = 0 \quad\text{para}\quad i\neq j $$

estas relaciones no pueden ser cumplidas por escalares, con lo que la forma más simple que puede satisfacer dichas relaciones son matrices. Como $\beta^2=I$ y la traza cumple la relación de cíclica Tr($ABC$)=Tr($BCA$), se tiene

$$ \text{Tr}(\alpha_i) = \text{Tr}(\alpha_i\beta\beta) = \text{Tr}(\beta\beta\alpha_i)=\text{Tr}(\beta\alpha_i\beta) = -\text{Tr}(\beta\beta\alpha_i) = -\text{Tr}(\alpha_i) $$

como $\text{Tr}(\alpha_i)=-\text{Tr}(\alpha_i)$ se tiene que $\text{Tr}(\alpha_i)=0$. De forma similar para $\beta$. Tomando la ecuación de autovalores y multiplicándola por $\alpha_i$

$$ \alpha_i X = \lambda X \quad \rightarrow \quad \alpha_i^2X=\lambda^2X$$

como $\alpha_i^2=I$, se tiene que $\lambda^2=1$, por lo que $\lambda=\pm 1$. Como la suma de los autovalores de una matriz es igual a su traza, la traza es igual a cero y los autovalores son $\lambda=\pm1$, se tiene que las matrices deben tener una dimensión par. Dado que para matrices 2x2 solo se tienen tres matrices anticonmutativas de traza cero, como lo pueden ser las matrices de Pauli, la representación más pequeña de las matrices es de dimensión 4x4. Por lo que la función de onda toma la forma de un vector de cuatro componentes, llamdo spinor de Dirac

$$ \psi = \begin{pmatrix}
    \psi_1 \\
    \psi_2 \\
    \psi_3 \\
    \psi_4 \\
\end{pmatrix} $$

La representación convencional de las matrices $\alpha_i$ y $\beta$ es dada mediante las matrices de Pauli

$$ \beta = \begin{pmatrix}
    I & 0 \\
    0 & -I \\
\end{pmatrix} \quad , \quad \alpha_i = \begin{pmatrix}
    0 & \sigma_i \\
    \sigma_i & 0 \\
\end{pmatrix} $$

donde $I$ es la matriz identidad y $\sigma_i$ las matrices de Pauli. De igual forma, es necesario recalcar que las descripciones físicas de la ecuación no dependen de la representación. Como el operador energía debe ser hermítico, se tiene a su vez que 

$$ \alpha_i^{\dagger} = \alpha_i \quad \text{y} \quad \beta^{\dagger} = \beta $$

volviendo a la relación 

\begin{equation}
    i\frac{\partial \psi}{\partial t} =  -i\alpha_x\partial_x\psi -i\alpha_y\partial_y\psi -i\alpha_z\partial_z\psi +\beta m \psi
    \label{de1}
\end{equation}

y tomando su complejo conjugado

\begin{equation}
    -i\frac{\partial \psi^{\dagger}}{\partial t} = i\partial_x\psi^{\dagger}\alpha_x^{\dagger} +i\partial_y\psi^{\dagger}\alpha_y^{\dagger} +i\partial_{x}\psi^{\dagger}\alpha_z^{\dagger} +m\psi^{\dagger}\beta^{\dagger}
    \label{de2}
\end{equation}

tomando $\psi^{\dagger}\times (\ref{de1})-(\ref{de2})\times \psi$

$$ \psi^{\dagger}(-i\alpha_x\partial_x\psi -i\alpha_y\partial_y\psi -i\alpha_z\partial_z\psi +\beta m \psi) - (i\partial_x\psi^{\dagger}\alpha_x^{\dagger} +i\partial_y\psi^{\dagger}\alpha_y^{\dagger} +i\partial_{x}\psi^{\dagger}\alpha_z^{\dagger} +m\psi^{\dagger}\beta^{\dagger})\psi = i(\psi^{\dagger}\partial_{t}\psi + \partial_t \psi^{\dagger}\psi) $$

$$ -i[(\psi^{\dagger}\alpha_x\partial_x\psi + \partial_x \psi^{\dagger}\alpha_x\psi) + (\psi^{\dagger}\alpha_y\partial_y\psi + \partial_y \psi^{\dagger}\alpha_y\psi) + (\psi^{\dagger}\alpha_z\partial_z\psi + \partial_z \psi^{\dagger}\alpha_z\psi)]=i(\psi^{\dagger}\partial_{t}\psi + \partial_t \psi^{\dagger}\psi) $$

$$ -i[\partial_x(\psi^{\dagger\alpha_x\psi}) + \partial_y(\psi^{\dagger\alpha_y\psi}) + \partial_z(\psi^{\dagger\alpha_z\psi})]=i(\psi^{\dagger}\partial_{t}\psi + \partial_t \psi^{\dagger}\psi) $$

$$ \boldsymbol\nabla \cdot (\psi^{\dagger}\boldsymbol\alpha\psi) = -\partial_t(\psi^{\dagger}\psi) $$

por comparación con la ecuación de continuidad

$$ \rho = \psi^{\dagger}\psi \quad ,\quad \textbf{j} = \psi^{\dagger}\boldsymbol\alpha\psi $$

por lo que la probabilidad $\rho = |\psi_1|^2 + |\psi_2|^2 +|\psi_3|^2+|\psi_4|^2 $ toma siempre valores positivos. Multiplicando la ecuación \ref{de1} por $\beta$ se tiene

$$ i\beta\partial_t \psi +i\beta\alpha_x\partial_x\psi + \beta\alpha_y\partial_y\psi \beta\alpha_z\partial_z\psi -\beta^2m\psi = 0 $$

rotulando $\beta = \gamma^0$, $\beta\alpha_i = \gamma^i$

$$ i\gamma^0\partial_t \psi + i\gamma^1\partial_x \psi + i\gamma^2\partial_y \psi + i\gamma^3\partial_z \psi -m\psi =0 $$

por lo que, expresando la ecuación de Dirac en la forma covariante

$$ (i\gamma^{\mu}\partial_{\mu} - m)\psi = 0 $$

\section{Solución de la ecuación de Dirac}

\subsection{Solución para una partícula en reposo}

Asumiendo una solución de onda plana de la forma

$$ \psi ( \textbf{x}, t) = u(E, \textbf{p})e^{i(\textbf{p}\cdot\textbf{x} - Et)} $$

donde $u(E, \textbf{p})$ es un 4-vector y la función $\psi$ satisface la ecuación de Dirac

$$ (i\gamma^{\mu}\partial_{\mu} -m)\psi(\textbf{x}, t)=0 $$

aplicando las derivadas

$$ i\gamma^{\mu}\partial_{\mu}[u(E, \textbf{p} ) e^{i(\textbf{p} \cdot \textbf{x} - Et)}] - mu(E, \textbf{p})e^{i(\textbf{p}\cdot\textbf{x} - Et)} = 0 $$

$$ \gamma^{\mu}p_{\mu}[u(E, \textbf{p} ) e^{i(\textbf{p} \cdot \textbf{x} - Et)}] = mu(E, \textbf{p})e^{i(\textbf{p}\cdot\textbf{x} - Et)} $$

Para una partícula en reposo se tiene

$$ \gamma^0 Eu(E, 0)e^{-iEt} = mu $$

$$ E\begin{pmatrix}
    1 & 0 & 0 & 0 \\
    0 & 1 & 0 & 0 \\
    0 & 0 & -1 & 0 \\
    0 & 0 & 0 & -1 \\
\end{pmatrix}\begin{pmatrix}
    \psi_1 \\
    \psi_2 \\
    \psi_3 \\
    \psi_4 \\
\end{pmatrix} = m\begin{pmatrix}
    \psi_1 \\
    \psi_2 \\
    \psi_3 \\
    \psi_4 \\
\end{pmatrix}  $$

$$ \begin{pmatrix}
    E\cdot\psi_1 \\
    E\cdot\psi_2 \\
    -E\cdot\psi_3 \\
    -E\cdot\psi_4 \\
\end{pmatrix} = \begin{pmatrix}
    m\cdot\psi_1 \\
    m\cdot\psi_2 \\
    m\cdot\psi_3 \\
    m\cdot\psi_4 \\
\end{pmatrix}  $$

por lo que hay dos soluciones con $E=m$ y dos soluciones con $E=-m$. Los spinores solución deben ser ortogonales:

$$ u_1(E,0) = N\begin{pmatrix}
    1 \\
    0 \\
    0 \\
    0 \\
\end{pmatrix} \quad , \quad u_2(E,0) = N\begin{pmatrix}
    0 \\
    1 \\
    0 \\
    0 \\
\end{pmatrix} $$

$$ u_3(E,0) = N\begin{pmatrix}
    0 \\
    0 \\
    1 \\
    0 \\
\end{pmatrix} \quad , \quad u_4(E,0) = N\begin{pmatrix}
    0 \\
    0 \\
    0 \\
    1 \\
\end{pmatrix} $$

donde $N$ es el factor de normalización de $u$. Finalmente, las solución, que incluye el factor temporal

$$ \psi_1(E,0) = N\begin{pmatrix}
1 \\
0 \\
0 \\
0 \\
\end{pmatrix}e^{-imt} \quad , \quad \psi_2(E,0) = N\begin{pmatrix}
0 \\
1 \\
0 \\
0 \\
\end{pmatrix}e^{-imt} \quad , \quad \psi_3(E,0) = N\begin{pmatrix}
0 \\
0 \\
1 \\
0 \\
\end{pmatrix}e^{imt} \quad , \quad	 \psi_4(E,0) = N\begin{pmatrix}
0 \\
0 \\
0 \\
1 \\
\end {pmatrix}e^{imt} $$

\subsection{Solución general}

Para el caso general de una partícula que se mueva a velocidad constante se tiene que, con la solución de onda plana planteada anteriormente

$$ (\gamma^0E - \gamma^1p_x - \gamma^2p_y - \gamma^3p_z - m)u(E,p)= 0 $$

asumiendo una solución de la forma $u=\begin{pmatrix} u_A \\ u_B \end{pmatrix}$, la ecuación se puede expresar de la forma

$$ \left(E\begin{pmatrix}
I & 0 \\
0 & -I \\
\end{pmatrix} - \begin{pmatrix}
0 & \boldsymbol\sigma \cdot \textbf{p} \\
-\boldsymbol\sigma \cdot \textbf{p} & 0 
\end{pmatrix} - m\begin{pmatrix}
I & 0 \\
0 & I \\
\end{pmatrix}\right)\begin{pmatrix}
u_A \\
u_B
\end{pmatrix} = 0 $$

$$ \begin{pmatrix}
(E-m)I & -\boldsymbol\sigma \cdot \textbf{p} \\
\boldsymbol\sigma \cdot \textbf{p} & -(E+m)I
\end{pmatrix}\begin{pmatrix}
u_A \\
u_B
\end{pmatrix}= 0 $$

de lo que resultan las dos ecuaciones

$$ u_A = \frac{\boldsymbol\sigma \cdot \textbf{p}}{E-m}u_B $$

$$ u_B = \frac{\boldsymbol\sigma \cdot \textbf{p}}{E+m}u_A $$

tomando la forma ortogonal más simple para $u_A$

$$ u_A = \begin{pmatrix}
1 \\
0
\end{pmatrix} \quad , \quad u_A = \begin{pmatrix}
0 \\
1
\end{pmatrix} $$

y teniendo

$$ \boldsymbol\sigma \cdot \textbf{p} = \begin{pmatrix}
p_z & p_x -ip_y \\
p_x+ip_y & -p_z 
\end{pmatrix} $$

por lo que las dos soluciones para $u_B$ 

$$ u_B = \frac{1}{E+m}\begin{pmatrix}
p_z \\
p_x + ip_y
\end{pmatrix} $$

$$ u_B = \frac{1}{E+m}\begin{pmatrix}
p_x - ip_y \\
-p_z
\end{pmatrix} $$

tomando ahora la forma ortogonal más simple para $u_B$

$$ u_B = \begin{pmatrix}
1 \\
0
\end{pmatrix} \quad , \quad u_B = \begin{pmatrix}
0 \\
1
\end{pmatrix} $$

por lo que las dos soluciones para $u_A$ 

$$ u_A = \frac{1}{E-m}\begin{pmatrix}
p_z \\
p_x + ip_y
\end{pmatrix} $$

$$ u_A = \frac{1}{E-m}\begin{pmatrix}
p_x - ip_y \\
-p_z
\end{pmatrix} $$

por lo que, finalmente, la solución con el factor exponencial es

$$ \psi_1(E,\textbf{p})=N_1\begin{pmatrix}
1 \\
0 \\
\frac{p_z}{E+m} \\
\frac{p_x +ip_y}{E+m} 
\end{pmatrix}e^{i(\textbf{p}\cdot \textbf{x}-Et)} \quad , \quad \psi_2(E,\textbf{p})=N_2\begin{pmatrix}
0 \\
1 \\
\frac{p_x-ip_y}{E+m} \\
\frac{-p_z}{E+m}
\end{pmatrix}e^{i(\textbf{p}\cdot \textbf{x}-Et)} $$

$$ \psi_3(E,\textbf{p})=N_3\begin{pmatrix}
\frac{p_z}{E-m} \\
\frac{p_x + ip_y}{E-m} \\
1 \\
0 
\end{pmatrix}e^{i(\textbf{p}\cdot \textbf{x}+Et)} \quad , \quad \psi_4(E,\textbf{p})=N_4\begin{pmatrix}
\frac{p_x-ip_y}{E-m} \\
\frac{-p_z}{E-m} \\
0 \\
1 
\end{pmatrix}e^{i(\textbf{p}\cdot \textbf{x}+Et)}  $$

donde $N_i$ son los factores de normalización de las funciones de onda.